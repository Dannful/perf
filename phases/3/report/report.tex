%%
%% This is file `sample-acmsmall.tex',
%% generated with the docstrip utility.
%%
%% The original source files were:
%%
%% samples.dtx  (with options: `all,journal,bibtex,acmsmall')
%% 
%% IMPORTANT NOTICE:
%% 
%% For the copyright see the source file.
%% 
%% Any modified versions of this file must be renamed
%% with new filenames distinct from sample-acmsmall.tex.
%% 
%% For distribution of the original source see the terms
%% for copying and modification in the file samples.dtx.
%% 
%% This generated file may be distributed as long as the
%% original source files, as listed above, are part of the
%% same distribution. (The sources need not necessarily be
%% in the same archive or directory.)
%%
%%
%% Commands for TeXCount
%TC:macro \cite [option:text,text]
%TC:macro \citep [option:text,text]
%TC:macro \citet [option:text,text]
%TC:envir table 0 1
%TC:envir table* 0 1
%TC:envir tabular [ignore] word
%TC:envir displaymath 0 word
%TC:envir math 0 word
%TC:envir comment 0 0
%%
%% The first command in your LaTeX source must be the \documentclass
%% command.
%%
%% For submission and review of your manuscript please change the
%% command to \documentclass[manuscript, screen, review]{acmart}.
%%
%% When submitting camera ready or to TAPS, please change the command
%% to \documentclass[sigconf]{acmart} or whichever template is required
%% for your publication.
%%
%%
\documentclass[acmsmall]{acmart}
\usepackage{makecell}
\usepackage[brazil]{babel}
%%
%% \BibTeX command to typeset BibTeX logo in the docs
\AtBeginDocument{%
  \providecommand\BibTeX{{%
    Bib\TeX}}}

%% Rights management information.  This information is sent to you
%% when you complete the rights form.  These commands have SAMPLE
%% values in them; it is your responsibility as an author to replace
%% the commands and values with those provided to you when you
%% complete the rights form.
\setcopyright{acmlicensed}
\copyrightyear{2025}
\acmYear{2025}

%%
%% These commands are for a JOURNAL article.
\acmJournal{JACM}
\acmVolume{37}
\acmNumber{4}
\acmArticle{111}
\acmMonth{11}

%%
%% Submission ID.
%% Use this when submitting an article to a sponsored event. You'll
%% receive a unique submission ID from the organizers
%% of the event, and this ID should be used as the parameter to this command.
%%\acmSubmissionID{123-A56-BU3}

%%
%% For managing citations, it is recommended to use bibliography
%% files in BibTeX format.
%%
%% You can then either use BibTeX with the ACM-Reference-Format style,
%% or BibLaTeX with the acmnumeric or acmauthoryear sytles, that include
%% support for advanced citation of software artefact from the
%% biblatex-software package, also separately available on CTAN.
%%
%% Look at the sample-*-biblatex.tex files for templates showcasing
%% the biblatex styles.
%%

%%
%% The majority of ACM publications use numbered citations and
%% references.  The command \citestyle{authoryear} switches to the
%% "author year" style.
%%
%% If you are preparing content for an event
%% sponsored by ACM SIGGRAPH, you must use the "author year" style of
%% citations and references.
%% Uncommenting
%% the next command will enable that style.
%%\citestyle{acmauthoryear}


%%
%% end of the preamble, start of the body of the document source.
\begin{document}

%%
%% The "title" command has an optional parameter,
%% allowing the author to define a "short title" to be used in page headers.
\title[Análise: Fletcher]{Análise de uma aplicação de propagação de ondas sísmicas: Fletcher}

%%
%% The "author" command and its associated commands are used to define
%% the authors and their affiliations.
%% Of note is the shared affiliation of the first two authors, and the
%% "authornote" and "authornotemark" commands
%% used to denote shared contribution to the research.
\author{Francisco Pegoraro Etcheverria}
\author{Pedro Colle}
\author{Vinícius Daniel Spadotto}
\affiliation{%
  \institution{Universidade Federal do Rio Grande do Sul}
  \city{Porto Alegre}
  \state{Rio Grande do Sul}
  \country{Brasil}
}

%% By default, the full list of authors will be used in the page
%% headers. Often, this list is too long, and will overlap
%% other information printed in the page headers. This command allows
%% the author to define a more concise list
%% of authors' names for this purpose.
\renewcommand{\shortauthors}{Etcheverria et al.}

%%
%% The abstract is a short summary of the work to be presented in the
%% article.
\begin{abstract}
  A clear and well-documented \LaTeX\ document is presented as an
  article formatted for publication by ACM in a conference proceedings
  or journal publication. Based on the ``acmart'' document class, this
  article presents and explains many of the common variations, as well
  as many of the formatting elements an author may use in the
  preparation of the documentation of their work.
\end{abstract}

%%
%% Keywords. The author(s) should pick words that accurately describe
%% the work being presented. Separate the keywords with commas.
\keywords{Onda, Sísmica, Propagação, Perturbação, Fletcher}

\received{30 November 2025}
\received[revised]{30 November 2025}
\received[accepted]{30 November 2025}

%%
%% This command processes the author and affiliation and title
%% information and builds the first part of the formatted document.
\maketitle

\section{Introduction}


\section{Trabalhos relacionados}
\section{Metodologia}

A metodologia empregada para a realização do presente trabalho deu-se mediante
a execução de experimentos com fatores pré-definidos e sob condições de execução estabelecidas e
devidamente controladas para atingir os objetivos estipulados.

\subsection{Objetivos}

Visou-se a configurar um ambiente suficientemente controlado para a observação do comportamento da aplicação,
focando em como os fatores definidos para a execução impactam nas saídas capturadas. Definiu-se, pois, como objetivo a
obtenção dos dados brutos e a subsequente utilização destes para a construção de um modelo de regressão linear que apropriadamente
relacionasse os fatores à saída---esta dada em \textit{MSamples/s}---devidamente fundamentada por um teste de hipótese adequado.

\subsection{Projetos experimentais}

Foram realizados \textbf{2} projetos experimentais \textit{full factorial} de múltiplos fatores em sequência, tendo o último sido baseado nas observações
advindas da execucação do primeiro.

Cada projeto foi dividido em dois conjuntos de experimentos distintos: \textbf{CPU} e \textbf{GPU}.

\begin{table}[H]
  \caption{Projetos experimentais para CPU}
  \label{tab:projexpcpu}
  \begin{tabular}{ccccc}
    \toprule
    Projeto experimental&Dimensões do dado de entrada&N\textdegree de iterações&N\textdegree de \textit{threads}&Replicações\\
    \midrule
    1 & $24^3$, $56^3$, $120^3$ & $10$, $100$, $1000$ & $1$, $2$, $4$, $8$, $16$ & 100\\
    2 & $24^3$, $88^3$, $152^3$, $248^3$, $376^3$, $504^3$ & $25$ & $1$, $2$, $4$, $8$, $16$ & 10\\
    \bottomrule
  \end{tabular}
\end{table}

\begin{table}[H]
  \caption{Projetos experimentais para GPU}
  \label{tab:projexpgpu}
  \begin{tabular}{cccc}
    \toprule
    Projeto experimental&Dimensões do dado de entrada&N\textdegree de iterações&Replicações\\
    \midrule
    1 & $24^3$, $56^3$, $120^3$ & $10$, $100$, $1000$ & 100\\
    2 & $24^3$, $88^3$, $152^3$, $248^3$, $376^3$, $504^3$ & $25$ & 10\\
    \bottomrule
  \end{tabular}
\end{table}

\subsection{Ambiente de execução}

Todos os experimentos foram executados no PCAD, https://gppd-hpc.inf.ufrgs.br, no INF/UFRGS, utilizando
as máquinas \textbf{draco} e \textbf{beagle}.

\begin{table}[H]
  \caption{Máquinas utilizadas para os experimentos}
  \label{tab:macexpcpu}
  \begin{tabular}{ccc}
    \toprule
    Projeto experimental&Experimento&Máquina\\
    \midrule
    1 & CPU & draco2\\
    1 & GPU & draco1\\
    2 & CPU & draco2\\
    2 & GPU & draco1\\
    2 & GPU & beagle\\
    \bottomrule
  \end{tabular}
\end{table}

\begin{table}[H]
  \caption{Especificações técnicas das máquinas utilizadas}
  \label{tab:macspec}
  \begin{tabular}{cccccc}
    \toprule
    Máquina&CPU&RAM&Acelerador&Disco\\
    \midrule
    draco[1,2] & \makecell{2 x Intel(R) Xeon(R) \\ E5-2640 v2, 2.00 GHz, \\ 32 threads, 16 cores} & 64 GB DDR3 & \makecell{NVIDIA Tesla K20m} & 1.8 TB HDD\\
    \hline
    beagle & \makecell{2 x Intel(R) Xeon(R) \\ E5-2650, 2.00 GHz, \\ 32 threads, 16 cores} & 32 GB DDR3 & \makecell{2 x NVIDIA GeForce GTX 1080 Ti} & 931 GB HDD\\
    \bottomrule
  \end{tabular}
\end{table}

\section{Resultados}

Após a execução do primeiro projeto experimental, observou-se a possibilidade de realizar um novo projeto,
almejando um tamanho de problema de entrada maior, porém com uma quantidade menor de iterações. Conforme
as tabelas \ref{tab:projexpcpu} e \ref{tab:projexpgpu}, a primeira bateria experimental confere uma variabilidade menor
às observações, em virtude da maior quantidade de replicações para cada execução. Entretanto, a maior diversidade de tamanho
de problema de entrada permite a identificação mais precisa da influência de tal fator na métrica de saída (\textit{MSamples/s}).

Os projetos experimentais supracitados permitiram a obtenção de um $R^2 \approx 0.9898$ para os experimentos de \textbf{CPU} e um
$R^2 \approx 0.68$ para \textbf{GPU}.

Embora a métrica $R^2$ para CPU seja satisfatória, o modelo revelou que o fator de maior contribuição para a métrica de saída é o
\textbf{número de threads}, sendo a razão para tal imediata: a complexidade de cálculo associada ao tamanho do problema manteve-se
similar em sua ordem de grandeza, implicando a situação supracitada, porquanto a aplicação comporta-se conforme $O(n)$ em relação às dimensões do dado
de entrada. A quantidade de iterações, por sua vez, embora tenha variado razoavelmente em sua magnitude, depende intrinsicamente da complexidade de cálculo
associada, levando, pois, à sua tênue contribuição para os valores observados.

Outrossim, o $R^2$ calculado para a GPU é esperado: dado que a complexidade de cálculo torna-se substancialmente menor em virtude da maior exploração
do paralelismo, o modelo de regressão dispõe de informações altamente limitadas para determinar como cada fator influencia a saída. Dessa forma, todos os fatores
acabam por contribuir de maneira similarmente igual e tênue para os valores de saída, conferindo elevada incerteza para o modelo.

%%
%% The acknowledgments section is defined using the "acks" environment
%% (and NOT an unnumbered section). This ensures the proper
%% identification of the section in the article metadata, and the
%% consistent spelling of the heading.
\begin{acks}
To Robert, for the bagels and explaining CMYK and color spaces.
\end{acks}

%%
%% The next two lines define the bibliography style to be used, and
%% the bibliography file.
\bibliographystyle{ACM-Reference-Format}
\bibliography{report}

\end{document}
\endinput
